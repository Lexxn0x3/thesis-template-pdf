\section{Problemdarstellung und Zielsetzung}
In großen Unternehmen und vor allem Fabriken kommen viele Accesspoints zum einsatz.
Immer mehr Industrielle Geräte verfügen nun auch über WiFi anbindungen, wodurch dieses Zuverlässig und Stabiel funktionieren muss.
Um dies zu gewärkstelligen müssen sämtliche Accesspoints richtig konfiguriert werden.
Das große Problem hierbei; sämtliche Accesspoints müssen einzeln behandelt werden; deren Konfiguration heruntergeladen, verglichen, angepasst und wieder hochgeladen werden.
Dies kostet sehr viel Zeit und erschwert ist ein Netzwerk mit vielen Accesspoints richtig und optimal zu konfigurieren, da es weder eine gesamtübersicht über alle Accesspoints gibt, weder alle Accesspoints gleichzeitig einheitlich Konfiguriert werden müssen.
Auch gibt es somit keine möglichkeit Zentral, offensichtliche fehlkonfigurationen zu erkennen.
\subsection{Interesse Siemens}
Meine Abteilung wünscht sich nun ein Tool, welches im Feld benutz werden kann um (Siemens-)Accesspoints in einem Netzwerk zu erkennen und all deren Konfigurationen zu laden und in einem Übersichtlichem Format darzustellen.
Die Darstellung sollte so gestaltet sein, dass es für einen Benutzer einfach ist Unterschiedliche Konfigurationen miteinander zu vergleichen und Anpassungen vorzunehmen.
Des weiteren können die Konfigurationen Analysiert werden, um gewisse Konfigurationsfehler zu erkennen und diese anzuzeigen.
Ein weiter wunsch von seiten meiner Abteilung ist es das ganze möglich Resourcenschonend durchzuführen.
Dazu zählt zum einen ein möglichst effizientes Programm, das schnell und einfach einsatz berreit ist, als auch ein möglichst geringer Netzwerk traffik, der von dem program produziert wird.
\subsection{Fachliche Herausvorderung}
Herausvorderungen wären zum einen eine effiziente Datenhaltung zu entwickeln um all die verschieden Konfigurationen zu speicher und anpassen zu können. 
Es gibt viele Eigenschaften, welche auf allen APs gleich sein sollten, wie \zb die SSIDs, andere Konfigurationen wie \zB Channels sollten unterschiedlich sein, wenn gewünscht. 
All dies müsste in der Datenstruktur abbildbar sein um gewisse Eigenschaften gleich zu halten und gewisse für jeden AP individuell.\\
\\
Damit einhergehend muss eine geignet visualisierung Entwickelt werden um zum einen die Topologie des Netzwerkes darstellen zu können, als auch Konfigurationen Übersichtlich vergleichbar zu machen, ohne manuel gewissen zeilen vergleichen zu müssen.\\
\\
Eine weiter Herausvorderung ist eine effizientes Design des Programms.
Zum einen ein recht Minimales und schnell einsazbereites Programm, als auch eine Minimierung von Traffik.
So wäre es \zB denkbar nach dem Auslesen der Konfiguration nur änderungen auf APs zu spielen und mit Caching zu arbeiten, so dass Nach einmaligem auslesen eines APs dieser kein zweites mal gelesen wird, sondern der letzte lokale stand verwendet wird. Hier wäre auch noch viel Potential sich weite\\

re tiefergehende optimierungen zu überlegen und zu erforschen und auf realisierbarkeit mit dem SNMP protokoll zu Überprüfen.
\section{Anforderung Siemens}
\subsection{Effizienz}
Als erwähnt, sollte das Programm möglichst effizient arbeiten.
\subsection{Verwendung von NOA und SNMP}
Meine Abteilung hätte gerne eine anbindung an eine bereits existierende NOA-App, mit welcher auch Konfigurationen von gewissen APs eingelesen werden können. Diese kann auch Fehler werfen, welche integriert werden können. 
Das ausrollen der Konfigurationen soll per SNMP passieren, da dies am Zuverlässigsten mit der Hardware Funktioniert.
\section{Konzept}
Die Folgenden Fragen sollen in dieser Arbeit beatwortet werden:
\begin{itemize}
  \item Wie sieht eine Effiziente und Flexible Datenstruktur auf um Konfigurationsdatein miteinander zu vergleichen und gewisse Paramter einheitlich oder absichtlich verschieden zu halten?
  \item Wie kann eine solche Datenstruktur auf Konfigurationsfehler überprüft werden und nach welchen Kriterien können diese erkannt werden?
  \item Wie können Konfigurationen möglichst anschaulich und übersichtlich einem User angezeigt werden, mit einem Fokus auf unterschiede und Konfigurationsfehler?
  \item Wie viel Networktraffik bedeutet das Konfigurieren von APs für ein durchschnittliches/große anzahl von APs?
  \item Was sind Methoden um den Traffik auf ein Minimum zu reduzieren und wie können diese umgesetzt werden?
\end{itemize}
Zunächst sollen diese Fragen erforscht werden, wodurch ein Konzept entsteht, nach welchem ein Prototyp implimmentiert werden kann.
Dieser soll dan getestet werden und dessen Resultate evaluiert.
\section{Gliederung}
\begin{enumerate}
  \item Einleitung und Problemstellung
  \item Stand der Forschung und technologischer Hintergrund
  \item Methodik und Konzeptentwicklung
  \item Implementierung des Tools
  \item Testphase und Evaluation
  \item Schlussfolgerungen und Ausblick
\end{enumerate}

